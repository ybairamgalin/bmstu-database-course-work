%----------------------- Преамбула -----------------------
\documentclass[ut8x, 14pt, oneside, a4paper]{extarticle}

\usepackage{extsizes} % Для добавления в параметры класса документа 14pt

% Для работы с несколькими языками и шрифтом Times New Roman по-умолчанию
\usepackage{lscape} %для ориентации альбомной
\usepackage[english,russian]{babel}
\usepackage{fontspec}
\setmainfont{Times New Roman}
% ГОСТовские настройки для полей и абзацев
\usepackage[left=30mm,right=10mm,top=19mm,bottom=20mm]{geometry}
\usepackage{misccorr}
\usepackage{indentfirst}
\usepackage{enumitem}
\setlength{\parindent}{1.25cm}
%\setlength{\parskip}{1em} % поменять
%\linespread{1.3}
\renewcommand{\baselinestretch}{1.5}
\setlist{nolistsep} % Отсутствие отступов между элементами \enumerate и \itemize

% Дополнительное окружения для подписей
\usepackage{array}
\newenvironment{signstabular}[1][1]{
	\renewcommand*{\arraystretch}{#1}
	\tabular
}{
	\endtabular
}

% Переопределение стандартных \section, \subsection, \subsubsection по ГОСТу;
% Переопределение их отступов до и после для 1.5 интервала во всем документе
\usepackage{titlesec}
\titleformat{\section}[block]
{\bfseries\normalsize\filcenter}{\thesection}{1em}{}

\titleformat{\subsection}[hang]
{\bfseries\normalsize}{\thesubsection}{1em}{}
\titlespacing\subsection{\parindent}{\parskip}{\parskip}

\titleformat{\subsubsection}[hang]
{\bfseries\normalsize}{\thesubsubsection}{1em}{}
\titlespacing\subsubsection{\parindent}{\parskip}{\parskip}

% Работа с изображениями и таблицами; переопределение названий по ГОСТу
\usepackage{caption}
\captionsetup[figure]{name={Рисунок},labelsep=endash}
\captionsetup[table]{singlelinecheck=false, labelsep=endash}

\usepackage{graphicx}

% Цвета для гиперссылок и листингов
\usepackage{color}

% Гиперссылки \toc с кликабельностью
\usepackage{hyperref}

\hypersetup{
	linktoc=all,
	linkcolor=black,
	colorlinks=true,
	urlcolor=black,
}

% Листинги
\setsansfont{Arial}
\setmonofont{Courier New}

\usepackage{color} % Цвета для гиперссылок и листингов
%\definecolor{comment}{rgb}{0,0.5,0}
%\definecolor{plain}{rgb}{0.2,0.2,0.2}
%\definecolor{string}{rgb}{0.91,0.45,0.32}
%\hypersetup{citecolor=blue}
\hypersetup{citecolor=black}

\usepackage{listings}
\lstset{
	basicstyle=\footnotesize\ttfamily,
	language=[Sharp]C, % Или другой ваш язык -- см. документацию пакета
	commentstyle=\color{comment},
	numbers=left,
	numberstyle=\tiny\color{black},
	numbersep=5pt,
	tabsize=4,
	extendedchars=\true,
	breaklines=true,
	keywordstyle=\color{blue},
	frame=b,
	stringstyle=\ttfamily\color{black}\ttfamily,
	showspaces=false,
	showtabs=false,
	xleftmargin=17pt,
	framexleftmargin=17pt,
	framexrightmargin=5pt,
	framexbottommargin=4pt,
	showstringspaces=false,
	inputencoding=utf8x,
	keepspaces=true
}

\DeclareCaptionLabelSeparator{line}{\ --\ }
\DeclareCaptionFont{white}{\color{white}}
\DeclareCaptionFormat{listing}{\colorbox[cmyk]{0.43,0.35,0.35,0.01}{\parbox{\textwidth}{\hspace{15pt}#1#2#3}}}
\captionsetup[lstlisting]{
	format=listing,
	labelfont=white,
	textfont=white,
	singlelinecheck=false,
	margin=0pt,
	font={bf,footnotesize},
	labelsep=line
}

\usepackage{ulem} % Нормальное нижнее подчеркивание
\usepackage{hhline} % Двойная горизонтальная линия в таблицах
\usepackage[figure,table]{totalcount} % Подсчет изображений, таблиц
\usepackage{rotating} % Поворот изображения вместе с названием
\usepackage{lastpage} % Для подсчета числа страниц

\makeatletter
\renewcommand\@biblabel[1]{#1.}
\makeatother

\usepackage{color}
\usepackage[cache=false, newfloat]{minted}
\newenvironment{code}{\captionsetup{type=listing}}{}
\SetupFloatingEnvironment{listing}{name=Листинг}

\usepackage{amsmath}

\usepackage{titlesec}
\usepackage{hyperref}

\titleclass{\subsubsubsection}{straight}[\subsection]

\newcounter{subsubsubsection}[subsubsection]
\renewcommand\thesubsubsubsection{\thesubsubsection.\arabic{subsubsubsection}}
\renewcommand\theparagraph{\thesubsubsubsection.\arabic{paragraph}} % optional; useful if paragraphs are to be numbered

\titleformat{\subsubsubsection}
{\normalfont\normalsize\bfseries}{\thesubsubsubsection}{1em}{}
\titlespacing*{\subsubsubsection}
{14mm}{3.25ex plus 1ex minus .2ex}{1.5ex plus .2ex}

\makeatletter
\renewcommand\paragraph{\@startsection{paragraph}{5}{\z@}%
	{3.25ex \@plus1ex \@minus.2ex}%
	{-1em}%
	{\normalfont\normalsize\bfseries}}
\renewcommand\subparagraph{\@startsection{subparagraph}{6}{\parindent}%
	{3.25ex \@plus1ex \@minus .2ex}%
	{-1em}%
	{\normalfont\normalsize\bfseries}}
\def\toclevel@subsubsubsection{4}
\def\toclevel@paragraph{5}
\def\toclevel@paragraph{6}
\def\l@subsubsubsection{\@dottedtocline{4}{7em}{4em}}
\def\l@paragraph{\@dottedtocline{5}{10em}{5em}}
\def\l@subparagraph{\@dottedtocline{6}{14em}{6em}}
\makeatother

\setcounter{secnumdepth}{4}
\setcounter{tocdepth}{4}

\usepackage{xcolor}
\usepackage{listings}

\usepackage{xparse}

\NewDocumentCommand{\codeword}{v}{%
	\tt{\textcolor{black}{#1}}%
}

\lstset{language=C,keywordstyle={\bfseries \color{black}}}
\captionsetup{labelsep=endash}

