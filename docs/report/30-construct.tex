 \section{Конструкторский раздел} \label{desing}

В данном разделе курсовой работы будет рассмотрен процесс проектирования базы данных для разрабатываемого приложения. Будут выделены ключевые этапы проектирования, подробно проанализированы действия в рамках ролевой модели, спроектирован триггер и описаны основные принципы проектирования приложения. Целью данного раздела будет создание базы данных, которая обеспечит стабильную работу приложения и удовлетворит потребности пользователей.


На основе выделенных ранее сущностей спроектированы следующие объекты базы данных.
\begin{enumerate}	
	\item Users --- содержит информацию об участнике и включает следующие поля:
	\begin{itemize}[label=---]
		\item user{\_}id --- уникальный идентификатор пользователя;
		\item token --- авторизационный токен пользователя;
		\item login --- логин пользователя;
		\item name --- имя пользователя;
		\item phone --- телефон пользователя;
		\item created{\_}at ---  метка времени первой авторизации.
	\end{itemize}
	
	\item Events --- содержит информацию о событиях и включает слудующие поля:
	\begin{itemize}[label=---]
		\item event{\_}id --- уникальный идентификатор события;
		\item name --- название события.
	\end{itemize}
	
	\item Requests --- содержит информацию заявках на участие и включает слудующие поля:
	\begin{itemize}[label=---]
		\item request{\_}id --- уникальный идентификатор заявки;
		\item event{\_}id --- идентификтор конференции, на которую подана заявка;
		\item user{\_}id --- пользователь, подавший заявку;
		\item description --- описание заяки;
		\item status --- статус заявки;
		\item created{\_}at --- метка времени создания заявки;
		\item updated{\_}at --- метка времени обновления заявки.
	\end{itemize}	
	
	\item Permissions --- содержит информацию о правах, доступных пользователям, и включает слудующие поля:
	\begin{itemize}[label=---]
		\item slug --- строковый идентификатор заявки;
		\item user{\_}id --- идентификатор пользователя
	\end{itemize}
	
	\item FileMeta --- содержит мета-информация о файлах и включает слудующие поля:
	\begin{itemize}[label=---]
		\item uuid --- уникальный идентификатор файла;
		\item source{\_}name --- изначальное имя файла;
		\item hash --- хеш файла;
		\item created{\_}at -- время загрузки файла.
	\end{itemize}
	
	\item File--- содержит информацию об улове участника и включает слудующие поля:
	\begin{itemize}[label=---]
		\item uuid --- уникальный идентификатор файла;
		\item binary{\_}data --- бинарные данные.
	\end{itemize}
	
\end{enumerate}

\subsection{Ролевая модель}

Ролевая модель предполагает наличие трех ролей: участника, модератора и администратора. Стоит отметить, что модератор обладает всеми правами участника, а администратор --- всеми правами модератора.

Для реализации поставленной выше задачи программа должна предоставлять следующие возможности всем пользователям:
\begin{itemize}[label=---]
	\item авторизация через sso;
	\item сеоздание, редактирование, удаление заявки на участие;
	\item просмотр списка мероприятий;
	\item добавление приложений к заявке;
	\item просмотр статуса своих заявок;
	\item общение через комментарии к заявке.
\end{itemize}

Модератор также имеет право:
\begin{itemize}[label=---]
	\item просмтаривать все заявки;
	\item создавать, редактировать, удалять мероприятия;
	\item одобрять и отклонять заявки;
	\item назначать и удалять модераторов, администраторов;
\end{itemize}

Администратор обладаем всеми правами модератора, но кроме того может начначать и удалять модераторо, других администраторов.

\subsection{Триггер}

Триггер — это хранимая процедура особого типа, которую пользователь не вызывает непосредственно, а исполнение которой обусловлено действием по изменению данных: добавлением, модификацией, удалением строки в заданной таблице.

В сущности <<Request>> имеется поле updated{\_}at --- время посоеднего изменения заявки. Для гарантии корректности данного поля и исключения ошибок при update - запросах разумно создать триггер, который будет проставлять данному полю текущее время после совершения операции update над записью.

\subsection{Проектиоврание приложения}

\begin{enumerate}
	\item слой бизнес логики --- обработка основной логики работы приложения;
	\item слой доступа к данным --- подключение к базе данных, отправка запросов, получение информации из базы данных;
	\item слой программного интерфеса --- обработка запросов от пользователей, делегирование выполнения соответвующим сервисам бизнес-логики.
\end{enumerate}

\subsection{Вывод}

В результате проектирования базы данных для разрабатываемого прило-
жения спроектированы сущности базы данных и их связи, ролевая модель и триггер.

\pagebreak
