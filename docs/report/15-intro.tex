\section*{ВВЕДЕНИЕ}
\addcontentsline{toc}{section}{ВВЕДЕНИЕ}

В современном мире огромное количество данных генерируется ежедневно, и их анализ становится все более важным для принятия обоснованных решений в различных областях науки и бизнеса. Термин <<Big Data>> определяется как данные огромных объемов и
разнообразия, а также методы их обработки, которые помогают в
распределенном масштабе анализировать информацию[1]. Обработка данных значительного объема требует использования специальных инструментов и знаний. Именно поэтому передача такого опыта имеет ключевое значение в современном образовании.

Для передачи знаний, а также обсуждения и решения актуальных задач в области обработки больших объемов данных в России проводятся следующие мероприятия[1]: вебинары, воркшопы, выставки, конгрессы, конференции. Ораганизация таких мероприятий связана с рассмотрением организаторами заявок от потенциальных участников. Автоматизация данного процесса может сократить нагрузку на организаторов, а также сделать процесс более понятным и прозрачным для участников.

Целью курсовой работы является проектирование и разработка базы данных мероприятий по направлению «BigData».

Для достижения поставленной цели, необходимо решить следующие задачи:
\begin{itemize}[label=---]
	\item определить необходимый функционал приложения, предоставляющего доступ к базе данных;
	\item выделить роли пользователей, а также формализовать данные;
	\item проанализировать системы управления базами данных и выбрать подходящую систему для хранения данных;
	\item спроектировать базу данных, описать ее сущности и связи, спроектировать триггер;
	\item реализовать интерфейс для доступа к базе данных.
\end{itemize}

Итогом работы станет приложение, предоставляющее интерфейс доступа к базе данных.
\pagebreak
